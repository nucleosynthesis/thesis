\chapter{Theory and Motivations}
\label{chap:theory}

\emph{The goal of particle physics throughout the 20th century has been to identify 
the most elemental constituents of matter and understand the nature of
the fundamental forces acting between them.}

Now will talk about each section.

\section{The Standard Model of Particle Physics}

The Standard Model (SM) in unparalleled in terms of success as 
a well tested, precision model of fundamental particle physics. 
The SM provides a description of the electromagnetic, weak-nuclear
and strong-nuclear interactions in quantum field theoretic prescriptions,
incorporating both both relativistic and quantum mechanical effects.

\subsection{Fundamental Matter Particles}
All of the known fundamental constituents of matter
are spin-$\frac{1}{2}$ fermions. 
The free-particle equation of motion for a spin-$\frac{1}{2}$ particle with mass $m$, 
given in Equation~\ref{eqn:Dirac}, was provided by Dirac~\cite{null}.
\begin{equation}
(i\gamma^{\mu}\partial_{\mu} - m)\psi = 0
\end{equation}
The matrices, $\gamma^{\mu},~\mu\in{0,1,2,3}$, of Equation~\ref{eqn:Dirac} are
defined by the anti commutator relation 
$\gamma^{\mu}\gamma^{\nu}+\gamma^{\mu}\gamma^{\nu} = 2\eta_{\mu\nu}I_{4}$ where
$\eta_{\mu\nu}$ is the flat space-time metric $(+,-,-,-)$ and $I_{4}$ is the $4\times4$
identity matrix.
The solutions, $\psi$, to Dirac's equation yield the particle and anti-particle
states which satisfy the relativistic expression, 
$E^{2} = \mathbf{p}\cdot\mathbf{p} + m^{2}$. Here, $p^{\mu} = (E,\mathbf{p})$ 
denotes the four-momentum of a 
massive particle with momentum $\mathbf{p}$ and energy $E$.

lightest are \ldots heaviest are \ldots most matter made of blah.
The strong 
 
A summary of the known fundamental fermions in their three generations is given 
in Table~\ref{tab:fermions}.
These are separated into those particles which
do (quarks) and do not (leptons) interact with the strong-nuclear force.
The quarks are not seen free in nature \ldots confinement
\begin{table}[htbp!]
\begin{tabular}{|l|l c|l c|l c| c|}
\hline 
	& \textbf{\Rmnum{1} } & & \textbf{\Rmnum{2}} & & \textbf{\Rmnum{3}} & & \textbf{Charge} \\
\hline
Leptons & electron & $e$ & muon & $\mu$ & tau  & $\tau$  & -1 \\
	& electron neutrino & $\nu_{e}$ & muon neutrino & $\nu_{\mu}$ & tau neutrino & $\nu_{\tau}$  & 0 \\
\hline
Quarks  & up 	& $u$ & charm 	& $c$ & top 	&$t$  & $+\frac{2}{3}$  \\
	& down 	& $d$ & strange & $s$ & bottom 	&$b$  & $-\frac{1}{3}$	\\
\hline
\end{tabular}
\caption{Fundamental fermions in the Standard Model. All of the fundamental 
fermions are spin-$\frac{1}{2}$ particles. The anti-fermion counterparts are not listed
here.}
\label{tab:fermions}
\end{table}

\subsection{Fundamental Forces}
say something about each of the four forces, EM, strong and weak, ranges,
strengths etc\ldots
% -------------------
The range of the interactions depends on the inverse of the masses of the gauge bosons
that mediate them. For that, the range of the Electromagnetic interaction, mediated
% --------------------------

The fundamental forces of nature are mediated by the exchange of gauge bosons
with total spin quantum number of 1.   
These gauge bosons arise from consideration of the symmetries which 
the relevant theory possesses. The theories of electromagnetism 
and the strong-nuclear force, Quantum Electro-dynamics (QED) and 
Quantum Chromo-dynamics (QCD) yield massless mediator bosons, the photon
and the gluons which are a direct consequence of the gauge invariance of those
theories. The mediators of the weak-nuclear force arise only through 
the unification of the weak and electromagnetic interactions and the mixing
of their gauge fields (See Section~\ref{sec:ewksymmetry}). 
The weak gauge bosons, $W^{\pm}$ and $Z$, unlike the photon and gluons 
have a finite mass which has been measured experimentally~\citep{combinedWmass,pdg}.
A summary of the fundamental gauge bosons of the Standard Model is given in 
Table~\ref{tab:bosons}. A quantum description of gravity is not included in the Standard Model.
This is a reasonable approximation as the strength of this interaction 
is much smaller than the other three.
\begin{table}[htbp!]
\begin{tabular}{|l|l c|c|c|}
\hline 
	& \textbf{Mediator Particle} & & \textbf{Charge} & \textbf{Mass (GeV)} \\
\hline
Electromagnetism & photon & $\gamma$ 			& 0 & 0   \\
\hline
Strong Nuclear   & gluon  & $g_{j},$ $j\in\left\{1,\cdots8\right\}$ 	& 0 & 0   \\
\hline
Weak Nuclear 	 &  &  $W^{+}$ & +1 & 80.39 \\
	 	 &  &  $W^{-}$ & -1 & 80.39 \\
	 	 &  &  $Z$     & 0  & 91.19 \\
\hline
\end{tabular}
\caption{Fundamental gauge bosons in the Standard Model.
All of the gauge-bosons are spin-1 particles.
The masses of the $W^{\pm}$ and $Z$ bosons are taken from 
Refs~\citep{combinedWmass} and~\citep{pdg} respectively.}
\label{tab:bosons}
\end{table}

\subsection{Electroweak Gauge Symmetry}
\label{sec:ewksymmetry}

Symmetries in nature are often found to relate to some underlying physical principle 
or fundamental law. It was first shown by Emmy Noether 
that for any physical system which can be described in the Lagrangian formalism,
any symmetry of the Lagrangian has an associated conserved quantity~\cite{null}.
In the context of dynamical quantum theories, the particular characteristics of 
particle interactions can be used to construct the appropriate Lagrangian 
by means of identifying the appropriate group of transformations under which 
the Lagrangian should be symmetric (invariant). 

One of the major achievements of the twentieth century in 
the development of the Standard Model was the unification of the electromagnetic 
and weak interactions. The original proposal, by Glashow in 1961, was
to construct a theory which incorporates the characteristics of 
the weak and electromagnetic interactions by associating them 
with a particular symmetry group.
The physical nature of electroweak interactions are encoded into a Lagrangian which 
is invariant under transformations of the group $\sutwol\times\uone$. 
This group has three generators for the $\sutwol$ group, $\frac{1}{2}\tau_{i}$ 
where $\tau_{i},~i\in \{1,2,3 \}$ are the Pauli-spin matrices, and one
additional generator for $\uone$. The associated gauge fields are 
$\hat{\mathbf{W}}_{\mu} = \left(\hat{W}_{\mu}^{1},\hat{W}_{\mu}^{2},\hat{W}_{\mu}^{3}\right)$
 and $\hat{B}_{\mu}$.
An example Lagrangian for electron-neutrino interactions is 
given in Equation~\ref{eqn:ewklagrangianeg}.
\begin{eqnarray}
\lagr_{\nu_{e},e} & =& \bar{\chi}_{L}\gamma^{\mu}\left[i\partial_{\mu} 
		   - g\frac{1}{2}\boldsymbol{\tau}\cdot\hat{\mathbf{W}_{\mu}}
		   - g^{\prime}\left(-\frac{1}{2}\right)\hat{B}_{\mu}\right] \chi_{L}
\nonumber \\
& &		   + \bar{e}_{R}\gamma^{\mu}\left[i\partial_{\mu} 
		   - g^{\prime}(-1)\hat{B}_{\mu}\right]{e}_{R}
		     -\frac{1}{4}
		     \hat{\mathbf{W}}_{\mu\nu}\cdot\hat{\mathbf{W}}^{\mu\nu} 
		     -\frac{1}{4}
		     \hat{B}_{\mu\nu}\hat{B}^{\mu\nu}
\label{eqn:ewklagrangianeg}
\end{eqnarray}
where the field tensors, $\hat{\mathbf{W}}_{\mu\nu}$ and $\hat{B}_{\mu\nu}$ given in 
Equations~\ref{eqn:wmunu} and~\ref{eqn:bmunu},
describe the kinematics of the gauge fields.
\begin{equation}
\hat{\mathbf{W}}_{\mu\nu} = \partial_{\mu}\hat{\mathbf{W}}_{\nu} - \partial_{\nu}\hat{\mathbf{W}}_{\mu} - g \hat{\mathbf{W}}_{\mu}\wedge\hat{\mathbf{W}}_{\nu}
\label{eqn:wmunu}
\end{equation}
\begin{equation}
\hat{B}_{\mu\nu} = \partial_{\mu}\hat{B}_{\nu} - \partial_{\nu}\hat{B}_{\mu}.
\label{eqn:bmunu}
\end{equation}

Experimentally, it has been verified that weak nuclear force explicitly violates
parity, that is transformations under spatial inversions $x\rightarrow -x$~\citep{null}.
A fermion field, $\psi$, can be projected into its left and right handed components 
$\psi_{L}$ and $\psi_{R}$ using the operators $\frac{1}{2}(1\mp\gamma^{5})$ respectively 
where $\gamma^{5} = \gamma^{0}\gamma^{1}\gamma^{2}\gamma^{3}$.  
As the weak nuclear force only interacts with left-handed fermions, 
right-handed components of the fermion fields remain invariant under $\sutwol$ transformations.
The right-handed component therefore does not appear in the Lagrangian since it interacts with
neither the electromagnetic or weak interactions.
Under the $\sutwol\times\uone$ group, 
the left handed components of the leptonic fermion fields, 
$\chi_{L}$ of Equation~\ref{eqn:ewklagrangianeg}, transform as a doublet
\begin{equation}
\chi_{L}  =   
\everymath{\displaystyle} \begin{pmatrix}
\nu_{e} \\ 
e
\end{pmatrix}_{L}
 \longrightarrow 
\exp(-i\boldsymbol\alpha\cdot\frac{\boldsymbol\tau}{2} - i\alpha) 
\begin{pmatrix}
\nu_{e} \\ 
e
\end{pmatrix}_{L}
\end{equation}
\label{eqn:doublettrans}
whereas the right-handed component of the electron field transforms as a singlet.  
\begin{equation}
e_{R}
 \longrightarrow 
\exp(-2i{\alpha}) 
e_{R}.
\end{equation}
The transformations are ``local'' in the sense that the coefficients  
$\boldsymbol{\alpha}$ and $\alpha$ are functions of space-time. 
To maintain the symmetry under local transformations of this type, the gauge fields
transform as follows, 
\begin{eqnarray}
\hat{\mathbf{W}}_{\mu} & 
 \longrightarrow & 
\hat{\mathbf{W}}_{\mu} - \frac{1}{g}\partial_{\mu}\boldsymbol{\alpha} 
	- \boldsymbol{\alpha}\wedge\hat{\mathbf{W}}_{\mu} \\
\hat{{B}}_{\mu} & 
 \longrightarrow & 
\hat{{B}}_{\mu} - \frac{1}{g^{\prime}}\partial_{\mu}{\alpha} 
\end{eqnarray}

The Lagrangian of Equation~\ref{eqn:ewklagrangianeg} contains no explicit terms which 
relate to the mass of the electron ($m_{e}$). Including the electron's mass directly would 
result in the addition of the term,
\begin{eqnarray}
-m_{e}\bar{e}e  &=& -m_{e}\bar{e}\left[\frac{1}{2}\left(1-\gamma^{5}\right) 
		    + \frac{1}{2}\left(1+\gamma^{5}\right)\right]e \nonumber \\
		&=& -m_{e}\left(\bar{e}_{R}e_{L} + \bar{e}_{L}e_{R}\right).
\end{eqnarray}
As $e_{L}$ transforms as a member of a doublet and $e_{R}$ as a singlet, 
the addition of this term to Equation~\ref{eqn:ewklagrangianeg} 
would break the symmetry of the Lagrangian which motivated its construction, 
namely under transformations of the $\sutwol$ group.

The physical electroweak boson fields, $W_{\mu}^{\pm}$, $Z_{\mu}$ and photon field, 
$A_{\mu}$, are obtained through a mixture of the electroweak gauge fields as,
\begin{eqnarray}
W_{\mu}^{\pm} & = &  \sqrt{\frac{1}{2}} \left( W_{\mu}^{1} \mp i W_{\mu}^{2} \right) \nonumber \\
Z_{\mu} &  = & \cos\theta_{w} W_{\mu}^{3} - \sin\theta_{w} B_{\mu} \nonumber \\
A_{\mu} &  = & \sin\theta_{w} W_{\mu}^{3} + \cos\theta_{w} B_{\mu},
\label{eqn:ewkbosons}
\end{eqnarray}
where the mixing angle, $\theta_{w} = \tan^{-1}{\frac{g^{\prime}}{g}}$, relates
the couplings of the weak neutral and electromagnetic interactions.
As expected, there is no term which corresponds to the mass of the photon, however,
the same is true for the $W$ and $Z$ bosons. The masses of the $W$ and $Z$ bosons, 
given in Table~\ref{tab:bosons}, have been measured experimentally and found to be 
non-zero. The inclusion of mass terms in Equation~\ref{eqn:ewklagrangianeg} would also
break the symmetry of the Lagrangian. Furthermore, it has been shown that the inclusion
of these mass terms results in a loss of re normalizability of the theory making it  
less effective at predicting observables such as cross-sections and decay rates.
Instead, these masses can be generated via a spontaneous, rather than explicit
symmetry breaking.

\subsection{Spontaneous Symmetry Breaking: The Higgs Mechanism}
$\approx 2-3$ pages

