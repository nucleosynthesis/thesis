\chapter{Theory and Motivations}
\label{chap:theory}

The goal of particle physics is to identify 
the most elemental constituents of matter and understand the nature of
the fundamental forces acting between them. In this chapter, a brief summary of 
the components of the Standard Model  will be given along with the motivation for
the search for the Standard Model Higgs boson. 
Section~\ref{sec:sm} introduces the mechanism by which mass is generated in the 
Standard Model and its relation to the SM Higgs boson is highlighted.
In Section~\ref{sec:smhiggs}, 
searches for, and indirect constraints on, the SM Higgs boson before the start up of 
the Large Hadron Collider (LHC) are discussed. The section concludes with 
how the Higgs boson can be produced and observed in proton-proton collisions at the LHC.

\section{The Standard Model of Particle Physics}
\label{sec:sm}

The Standard Model (SM) is  
a well tested, precision model of particle physics. 
Within the confines of quantum field theory (QFT),  
the SM provides a description of the electromagnetic, weak-nuclear
and strong-nuclear interactions, incorporating
both relativistic and quantum mechanical effects.

\subsection{Fundamental Matter Particles}
All of the known fundamental constituents of matter
are spin-$\frac{1}{2}$ fermions. 
The equation of motion for a spin-$\frac{1}{2}$ particle with mass $m$, 
given in Equation~\ref{eqn:Dirac}, was provided by Dirac~\citep{wu}.
\begin{equation}
(i\gamma^{\mu}\partial_{\mu} - m)\psi = 0
\label{eqn:Dirac}
\end{equation}
The matrices $\gamma^{\mu}$, $\mu\in{0,1,2,3}$,  are
defined by the anti commutator relation 
$\gamma^{\mu}\gamma^{\nu}+\gamma^{\mu}\gamma^{\nu} = 2\eta_{\mu\nu}I_{4}$ where
$\eta_{\mu\nu}$ is the flat space-time metric $(+,-,-,-)$ and $I_{4}$ is the $4\times4$
identity matrix.
The solutions, $\psi$, to Equation~\ref{eqn:Dirac} yield the particle and anti-particle
states which satisfy the relativistic expression, 
$E^{2} = \mathbf{p}\cdot\mathbf{p} + m^{2}$, for a 
massive particle with momentum $\mathbf{p}$ and energy $E$.
 
The fundamental fermions are separated into those which
do (quarks) and do not (leptons) interact with the strong-nuclear force.
Quarks and leptons are grouped into three generations which share the same properties
but increase in mass. Unlike the leptons, quarks are not seen as free particles in nature, 
but rather are confined to exist within baryons composed of three quarks 
and quark-anti-quark pairs known as mesons.
A summary of the known fundamental fermions in their three generations is given 
in Table~\ref{tab:fermions}. 
\begin{table}[htbp!]
\begin{tabular}{|l|l c|l c|l c| c|}
\hline 
	& \textbf{\Rmnum{1} } & & \textbf{\Rmnum{2}} & & \textbf{\Rmnum{3}} & & \textbf{Charge} \\
\hline
Leptons & electron & $e$ & muon & $\mu$ & tau  & $\tau$  & -1 \\
	& electron neutrino & $\nu_{e}$ & muon neutrino & $\nu_{\mu}$ & tau neutrino & $\nu_{\tau}$  & 0 \\
\hline
Quarks  & up 	& $u$ & charm 	& $c$ & top 	&$t$  & $+\frac{2}{3}$  \\
	& down 	& $d$ & strange & $s$ & bottom 	&$b$  & $-\frac{1}{3}$	\\
\hline
\end{tabular}
\caption{Fundamental fermions in the Standard Model. All of the fundamental 
fermions are spin-$\frac{1}{2}$ particles. The anti-fermion counterparts are not listed
here.}
\label{tab:fermions}
\end{table}

\subsection{Fundamental Forces}

The fundamental forces of nature are mediated by the exchange of gauge bosons.
They are all spin-1 particles which arise from 
consideration of the symmetries which the relevant theory possesses 
(See Section~\ref{sec:ewksymmetry}). 
The quantum field theories of electromagnetism, Quantum Electro-dynamics (QED),
and the strong-nuclear force, Quantum Chromo-dynamics (QCD),
yield massless mediator bosons, the photon
and the gluons which are a direct consequence of the gauge invariance of those
theories. Despite this, the typical ranges over which the two interactions occur
are dramatically different; strong interaction effects are only apparent 
on a scale of around $10^{-15}$m whereas the range of electromagnetic interactions are effectively infinite.

The mediators of the weak-nuclear and electromagnetic forces arise through 
the unification of the theories of weak and electromagnetic interactions and the mixing
of the associated gauge fields. 
The weak gauge bosons, $W^{\pm}$ and $Z$, unlike the photon and gluons 
have a finite mass which has been measured experimentally~\citep{combinedWmass,pdg}.
A summary of the fundamental gauge bosons of the Standard Model is given in 
Table~\ref{tab:bosons}. A quantum description of gravity is not included in the Standard Model.
This is a reasonable approximation as the strength of this interaction 
is much smaller than the other three, thereby having no impact on the predictive power of the model.
\begin{table}[htbp!]
\begin{tabular}{|l|l c|c|c|}
\hline 
	& \textbf{Mediator Particle} & & \textbf{Charge} & \textbf{Mass (GeV)} \\
\hline
Electromagnetism & photon & $\gamma$ 			& 0 & 0   \\
\hline
Strong Nuclear   & gluon  & $g_{j},$ $j\in\left\{1,\cdots8\right\}$ 	& 0 & 0   \\
\hline
Weak Nuclear 	 &  &  $W^{+}$ & +1 & 80.39 \\
	 	 &  &  $W^{-}$ & -1 & 80.39 \\
	 	 &  &  $Z$     & 0  & 91.19 \\
\hline
\end{tabular}
\caption{Fundamental gauge bosons in the Standard Model.
All of the gauge-bosons are spin-1 particles.
The masses of the $W^{\pm}$ and $Z$ bosons are taken from 
References~\citep{combinedWmass} and~\citep{pdg} respectively.}
\label{tab:bosons}
\end{table}

\subsection{Electroweak Gauge Symmetry}
\label{sec:ewksymmetry}

Symmetries in nature are often found to relate to some underlying physical principle 
or fundamental law. It was first shown by Emmy Noether 
that for any physical system which can be described in the Lagrangian formalism,
any symmetry of the Lagrangian has an associated conserved quantity~\cite{noether}.
In the context of dynamical quantum theories, the particular characteristics of 
particle interactions can be used to construct the appropriate Lagrangian 
by means of identifying a particular group of transformations under which 
the Lagrangian should be symmetric (invariant). 

One of the major achievements of the twentieth century in 
the development of the Standard Model was the unification of the electromagnetic 
and weak interactions~\citep{glashow,weinberg,salam}. 
The original proposal, by Glashow in 1961, was
to construct a theory which incorporates the characteristics of 
the weak and electromagnetic interactions by associating them 
with a particular symmetry group~\citep{glashow}.
The physical nature of electroweak interactions is encoded into a Lagrangian which 
is invariant under transformations of the group $\sutwol\times\uone$. 
This group has three generators for $\sutwol$, $T_{i} = \frac{1}{2}\tau_{i}$ 
where $\tau_{i},~i\in \{1,2,3 \}$ are the $2\times2$ Pauli-spin matrices, and one
additional generator for $\uone$, $Y$.
The quantum numbers associated with the $\sutwol$ group, weak isospin $t_{1,2,3}$, and 
$\uone$ group, hypercharge $y$, are related to the electric charge $Q$ as,
\begin{equation}
Q = t_{3}+\frac{y}{2},
\end{equation} 
where the factor of $\frac{\displaystyle 1}{\displaystyle 2}$ is chosen by convention.
The associated gauge fields are 
$\hat{\mathbf{W}}_{\mu} = \left(\hat{W}_{\mu}^{1},\hat{W}_{\mu}^{2},\hat{W}_{\mu}^{3}\right)$
 and $\hat{B}_{\mu}$.
An example Lagrangian for interactions within the first leptonic generation of fermions, $\lagr_{G}$, is 
given in Equation~\ref{eqn:ewklagrangianeg}.
\begin{eqnarray}
\lagr_{G} & =& \bar{\chi}_{L}\gamma^{\mu}\left[i\partial_{\mu} 
		   - g\frac{1}{2}\boldsymbol{\tau}\cdot\hat{\mathbf{W}_{\mu}}
		   - g^{\prime}\left(-\frac{1}{2}\right)\hat{B}_{\mu}\right] \chi_{L}
\nonumber \\
& &		   + \bar{e}_{R}\gamma^{\mu}\left[i\partial_{\mu} 
		   - g^{\prime}(-1)\hat{B}_{\mu}\right]{e}_{R}
		     -\frac{1}{4}
		     \hat{\mathbf{W}}_{\mu\nu}\cdot\hat{\mathbf{W}}^{\mu\nu} 
		     -\frac{1}{4}
		     \hat{B}_{\mu\nu}\hat{B}^{\mu\nu}
\label{eqn:ewklagrangianeg}
\end{eqnarray}
where the bar notation denotes the adjoint of the field, $\bar{\psi}=\psi^{\dagger}\gamma^{0}$.
The field tensors, $\hat{\mathbf{W}}_{\mu\nu}$ and $\hat{B}_{\mu\nu}$ given in 
Equations~\ref{eqn:wmunu} and~\ref{eqn:bmunu},
describe the kinematics of the gauge fields.
\begin{equation}
\hat{\mathbf{W}}_{\mu\nu} = \partial_{\mu}\hat{\mathbf{W}}_{\nu} - \partial_{\nu}\hat{\mathbf{W}}_{\mu} - g \hat{\mathbf{W}}_{\mu}\wedge\hat{\mathbf{W}}_{\nu}
\label{eqn:wmunu}
\end{equation}
\begin{equation}
\hat{B}_{\mu\nu} = \partial_{\mu}\hat{B}_{\nu} - \partial_{\nu}\hat{B}_{\mu}.
\label{eqn:bmunu}
\end{equation}

Experimentally, it has been verified that the weak nuclear force explicitly violates
parity, that is transformations under spatial inversions $x\rightarrow -x$~\citep{wu}.
A fermionic field, $\psi$, can be projected into its left and right handed components, 
$\psi_{L}$ and $\psi_{R}$, using the operators $\frac{1}{2}(1\mp\gamma^{5})$ respectively 
where $\gamma^{5} = \gamma^{0}\gamma^{1}\gamma^{2}\gamma^{3}$. 
As the weak nuclear force only interacts with left-handed fermions, 
right-handed components of the fermion fields remain invariant under $\sutwol$ transformations.
The right-handed component of the neutrino field therefore does not appear 
in the Lagrangian, $\lagr_{G}$, since it interacts with neither the electromagnetic nor the weak interactions.
Under the $\sutwol\times\uone$ group, 
the left handed components of the leptonic fermion fields,
$\chi_{L}$ of Equation~\ref{eqn:ewklagrangianeg}, transform as a doublet
\begin{equation}
\chi_{L}  =   
\everymath{\displaystyle} \begin{pmatrix}
\nu_{e} \\ 
e
\end{pmatrix}_{L}
 \longrightarrow 
\exp(-i\boldsymbol\alpha\cdot\frac{\boldsymbol\tau}{2} - i\alpha) 
\begin{pmatrix}
\nu_{e} \\ 
e
\end{pmatrix}_{L}
\end{equation}
\label{eqn:doublettrans}
whereas the right-handed component of the electron field transforms as a singlet.  
\begin{equation}
e_{R}
 \longrightarrow 
\exp(-2i{\alpha}) 
e_{R}.
\end{equation}
The transformations are ``local'' in the sense that the coefficients  
$\boldsymbol{\alpha}$ and $\alpha$ are functions of space-time. 
To maintain the symmetry under local transformations of this type, the gauge fields
transform as follows, 
\begin{eqnarray}
\hat{\mathbf{W}}_{\mu} & 
 \longrightarrow & 
\hat{\mathbf{W}}_{\mu} - \frac{1}{g}\partial_{\mu}\boldsymbol{\alpha} 
	- \boldsymbol{\alpha}\wedge\hat{\mathbf{W}}_{\mu} \\
\hat{{B}}_{\mu} & 
 \longrightarrow & 
\hat{{B}}_{\mu} - \frac{1}{g^{\prime}}\partial_{\mu}{\alpha} 
\end{eqnarray}

The Lagrangian of Equation~\ref{eqn:ewklagrangianeg} contains no explicit terms which 
relate to the mass of the electron ($m_{e}$). Including the electron's mass directly would 
require the addition of the term,
\begin{eqnarray}
-m_{e}\bar{e}e  &=& -m_{e}\bar{e}\left[\frac{1}{2}\left(1-\gamma^{5}\right) 
		    + \frac{1}{2}\left(1+\gamma^{5}\right)\right]e \nonumber \\
		&=& -m_{e}\left(\bar{e}_{R}e_{L} + \bar{e}_{L}e_{R}\right).
\end{eqnarray}
As $e_{L}$ transforms as a member of a doublet and $e_{R}$ as a singlet, 
the addition of this term to Equation~\ref{eqn:ewklagrangianeg} 
would break the symmetry of the Lagrangian which motivated its construction, 
namely transformations under the $\sutwol$ group~\citep{aitchison}.

The physical electroweak boson fields, $\hat{W}_{\mu}^{\pm}$, $\hat{Z}_{\mu}$ and photon field, 
$\hat{A}_{\mu}$, are obtained through a mixture of the electroweak gauge fields as,
\begin{eqnarray}
\hat{W}_{\mu}^{\pm} & = &  \sqrt{\frac{1}{2}} \left( \hat{W}_{\mu}^{1} \mp i \hat{W}_{\mu}^{2} \right) \nonumber \\
\hat{Z}_{\mu} &  = & \cos\theta_{w} \hat{W}_{\mu}^{3} - \sin\theta_{w} \hat{B}_{\mu} \nonumber \\
\hat{A}_{\mu} &  = & \sin\theta_{w} \hat{W}_{\mu}^{3} + \cos\theta_{w} \hat{B}_{\mu},
\label{eqn:ewkbosons}
\end{eqnarray}
where the mixing angle, $\theta_{w} = \tan^{-1}{\frac{g^{\prime}}{g}}$, relates
the couplings of the weak neutral and electromagnetic interactions.
As expected, there is no term which corresponds to the mass of the photon, however,
the same is true for the $W$ and $Z$ bosons. The masses of the $W$ and $Z$ bosons, 
given in Table~\ref{tab:bosons}, have been measured experimentally and found to be 
non-zero. The inclusion of mass terms for these bosons in Equation~\ref{eqn:ewklagrangianeg} 
would also break the symmetry of the Lagrangian. Furthermore, it has been shown that the inclusion
of these mass terms results in a loss of re-normalizability of the theory, making it  
less effective at predicting observables such as cross-sections and decay rates~\citep{halzen}.
Instead, these masses can be generated via a spontaneous, rather than explicit, 
breaking of the symmetry.

\subsection{Spontaneous Symmetry Breaking: The Higgs Mechanism}

In quantum field theory, a symmetry is ``spontaneously'' broken when the Lagrangian
itself remains invariant while the vacuum state, for which the Hamiltonian of the theory
attains its minimum, does not~\cite{aitchison}. In the context of the electroweak theory, 
spontaneous symmetry breaking is
achieved through the introduction of a complex scalar field which attains a non-zero 
vacuum expectation value (VEV)~\citep{Higgs:1964ia,PhysRev.155.1554,Higgs:1964pj,Guralnik:1964eu,PhysRev.145.1156}. 
This field is an $SU(2)$ doublet,
\begin{equation}
\phi = 
\everymath{\displaystyle} \begin{pmatrix}
\phi^{+} \\ 
\phi^{0}
\end{pmatrix}.
\end{equation}
The Lagrangian, $\lagr_{G}$, of Equation~\ref{eqn:ewklagrangianeg} is modified to include
an additional term which is $\sutwol\times\uone$ invariant, $\lagr_{\phi}$
given by, 
\begin{equation}
\lagr_{\phi}=(\hat{D}_{\mu}\phi)^{\dagger}(\hat{D}^{\mu}\phi)  
	    +\mu^{2} \phi^{\dagger}\phi - \frac{\lambda}{4}(\phi^{\dagger}\phi)^{2},
\label{eqn:higgslagr}
\end{equation}
where the covariant derivative $\hat{D}^{\mu}$ which acts on $\phi$ is given by,
\begin{equation}
\hat{D}^{\mu} = \partial^{\mu} + ig\frac{1}{2}\boldsymbol{\tau}\cdot\hat{\mathbf{W}}^{\mu}
		+ ig^{\prime}\frac{1}{2} \hat{B}^{\mu}.
\end{equation}
The second two terms in Equation~\ref{eqn:higgslagr} correspond to the Higgs potential. 
In order to generate masses for the gauge bosons, the parameters, $\mu$ and $\lambda$,
must satisfy $\mu^{2}<0$ and $\lambda>0$. The choice of non-zero 
VEV must then be made so that only the $W$ and $Z$ bosons acquire mass, while the 
symmetry associated with electromagnetism remains unbroken, leaving the photon massless.
The choice suggested by Weinberg in 1967~\citep{weinberg} was,
\begin{equation}
\mathrm{VEV} = \langle0|\phi|0\rangle = 
\everymath{\displaystyle} \begin{pmatrix}
0 \\ 
\frac{v}{\sqrt{2}}
\end{pmatrix},
\end{equation}
where $v= \frac{\displaystyle 2\mu}{\displaystyle \sqrt{\lambda}}$. In order to 
obtain the physical particle spectrum, perturbations around the vacuum state
are considered. If $\boldsymbol{\hat{\theta}}$ and $\hat{H}$ represent small variations in 
the four degrees of freedom of the field $\phi$ then, 
\begin{equation}
\phi = \exp(-i\boldsymbol{\hat{\theta}}\cdot\frac{1}{2v}\boldsymbol{\tau}) 
\everymath{\displaystyle} \begin{pmatrix}
0 \\ 
\frac{1}{\sqrt{2}}(v+\hat{H})
\end{pmatrix}.
\end{equation}
This can be simplified by choosing the phase fields $\boldsymbol{\hat{\theta}}$ 
to be zero. The Lagrangian obtained by inserting $\phi$ with this form 
into Equation~\ref{eqn:higgslagr} and adding it to the Lagrangian of 
Equation~\ref{eqn:ewklagrangianeg} is,

\begin{eqnarray}
\lagr_{\phi}+\lagr_{G} 
	     & = &   \frac{1}{2} \partial_{\mu} \hat{H} \partial^{\mu} \hat{H} 
	     - \mu^{2} \hat{H}^{2} 
	     + \frac{1}{8} g^{2}v^{2}\hat{W}_{1\mu}\hat{W}_{1}^{\mu}
	     + \frac{1}{8} g^{2}v^{2}\hat{W}_{2\mu}\hat{W}_{2}^{\mu}	
\nonumber \\
	    & & - \frac{v^{2}}{8} \left(g^{2}+{g^{\prime}}^{2}\right)\hat{Z}_{\mu}\hat{Z}^{\mu}
	     + KB ,
\label{eqn:lagrssb}
\end{eqnarray}
where only terms which are at most second order in the fields are kept, illustrating the physical 
particle spectrum, and the 
fermion fields are dropped altogether. 
The relation between the $\hat{W}^{\mu}_{3}$ 
and $\hat{B}^{\mu}$ fields from Equation~\ref{eqn:ewkbosons} has been used to obtain
the physical photon, $\hat{A}^{\mu}$, and $\hat{Z}^{\mu}$ 
fields. 
From this form of the Lagrangian, it is clear that the $\hat{W}_{1}^{\mu}$,  
$\hat{W}^{\mu}_{2}$ and $\hat{Z}$ fields acquire mass. As the $\hat{W}_{1}^{\mu}$ and 
$\hat{W}^{\mu}_{2}$ fields mix to form the physical $\hat{W}^{\pm}$ fields, 
the $W^{\pm}$ bosons acquire a mass of $m_{W} = \frac{\displaystyle gv}{\displaystyle 2}$.
The mass of the $Z$ boson is given by 
$m_{Z}=\frac{\displaystyle 1}{\displaystyle 2}\sqrt{g^{2}+{g^{\prime}}^{2}}$, 
while there is no term associated with the mass of the photon.
An additional scalar field, $\hat{H}$ (the Higgs boson), 
remains in the Lagrangian with mass $\sqrt{2}\mu$. 
The term $KB$ in Equation~\ref{eqn:lagrssb} denotes additional kinetic terms for the $\hat{W}^{\mu}_{1}$, 
$\hat{W}^{\mu}_{2}$, $\hat{Z}^{\mu}$ and $\hat{A^{\mu}}$ fields. 
The masses of the fermions are generated by adding Yukawa coupling
terms,
\begin{equation} 
-\lambda_{f} \bar{\chi}_{L} \phi \psi_{R} + \lambda_{f^{\prime}}\bar{\psi}_{R}(-i\tau_{2}\phi^{*})\chi_{L},
\end{equation}
to $\lagr_{\phi}$. 
The couplings $\lambda_{f}$, $f=u,d,e,\mu\cdots$,  are directly related
to the mass of the fermions, specifically 
$\lambda_{f}\propto m_{f}$ such that the heavier fermions have stronger
coupling to the Higgs boson. Although the SM does not predict the values of these
couplings, the masses of the fermions are experimentally measurable allowing 
access to, and providing constraints on, the properties of the Higgs boson.

