\chapter{Conclusions and Outlook}
\label{chap:conclusions}

The Standard Model of particle physics provides the most precise description of 
fundamental physics and remains  the most experimentally verified
model available. The mechanism by which electroweak symmetry breaking occurs 
in the standard model, giving rise to the masses of the fundamental fermions and
bosons, predicts the existence of a new massive scalar boson, the Higgs boson.
Such a particle should be experimentally observable, although prior to the LHC being
turned on, no such particle had been discovered.

In this thesis, a search for this particle in proton-proton collisions recorded at the CMS detector 
has been described, through its decay to two photons. The decay channel, despite having a 
relatively low branching ratio, is one of the most sensitive at CMS due to the high resolution 
of the electromagnetic calorimeter and the narrow invariant mass peak it provides.
The analysis detailed employed the use of several multivariate analysis techniques
in order to provide the greatest sensitivity to a potential signal.
As the signal yield in the two photon decay channel is small, 
the search for $\Hgg$ is highly sensitive to the background modelling.
The signal extraction technique described in this thesis was one which was 
developed by the author and served as a cross-check of the published result from the 2011 dataset.
This allowed for additional scrutiny on the background modelling 
to which the search in this decay channel is so sensitive.

In order to maximise the sensitivity of the search for the Standard Model Higgs boson,
data from several decay channels are combined at CMS using the methods described in this thesis.
An excess was observed in the combined data which is 
compatible with a Standard Model Higgs boson with a mass of 125 GeV. The excess was 
significant enough so as to claim discovery at the $5\sigma$ level. The excess is driven 
by the $\Hgg$ and  $\Hzzl$ channels, indicating that the particle is a boson; it has integer
spin. With the data available by the Hadron Collider Physics symposium of November 2012,
study of its couplings to Standard Model particles indicates that the new particle is 
consistent with the Standard Model Higgs boson, though additional data are required to 
make a definitive statement.

The discovery of the new particle is one of great significance to particle physics.
Should the particle turn out to conform to the predictions of the Standard Model, 
its discovery will have provided a great step into understanding the nature of 
electroweak symmetry breaking. However, if this turns out not to be the case, 
deviations from the predictions will indicate hints of potential new physics
and serve as guidance in the search for physics beyond the Standard Model . 
Additional data will be taken once the LHC resumes 
collisions in 2015 with an increased centre-of-mass energy, $\sqrt{s}=14$ TeV. 
With the additional data, stronger statements can be made as to 
the exact nature of the new particle and the energy scale at which the Standard Model 
must be retired. With this discovery in hand and the search potentially at an end, 
it is clear that a new window into fundamental physics has been opened, and the real 
work has only just begun.




