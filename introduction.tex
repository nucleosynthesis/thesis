\chapter{Introduction}
\label{chap:introduction}

The discovery of a new particle was announced by the ATLAS and CMS
Collaborations on the 4th of July 2012. The long-awaited discovery 
followed decades of experimental endeavours in the search for the
Higgs boson, the missing piece of the Standard Model (SM) of particle physics.
If further measurements of the properties of the new particle fit the SM predictions, 
the discovery will serve as compelling evidence for the mechanism by which
spontaneous symmetry breaking in the SM occurs, giving rise
to the masses of the fundamental fermions and bosons. 

In Chapter~\ref{chap:theory}, an introduction to the fundamental constituents of
matter and the interactions between them is given. The mechanism by which 
the fundamental fermions and bosons acquire mass in the SM, spontaneous symmetry breaking,
is outlined, serving as a motivation for the search for the SM Higgs boson. 
Previous searches and indirect constraints are discussed with the chapter concluding in
the search strategies employed at the LHC.

Chapter~\ref{chap:detector} describes the experimental apparatus required to undertake 
such a search, in particular the CMS detector which was used to collect the data upon which 
the majority of the author's research was conducted. This chapter includes a section describing 
a set of jet energy calibrations derived by the author which were subsequently used  
in the Level-1 trigger system at CMS.

The main analysis conducted by the author is detailed in Chapter~\ref{chap:hgg}. This chapter 
contains a description of the search for the Standard Model Higgs boson in the two
photon decay channel carried out on proton-proton collision data collected at CMS during 2011.
The focus of the chapter is on the background modelling technique developed by the author
used for statistical interpretations of the data. This method was one of two developed at CMS, 
which served as a cross-check of the background model used for the published result. 
The template signal modelling technique developed for this analysis is also used regularly by the $\Hgg$
working group at CMS for fast production of results and analysis development in a common analysis framework.
The chapter concludes with the updates for the 2012 analysis including data
collected at a centre of mass energy of 8 TeV. 

Finally, in Chapter~\ref{chap:combinations}, the statistical tools employed and developed
at CMS for the purposes of combined Higgs boson searches are detailed. The chapter includes 
the results presented at the July 2012 International Conference of High Energy Physics during which 
the announcement of the discovery of the new particle was made by the ATLAS and CMS Collaborations.
The section concludes with a discussion of the ongoing research at CMS intended to ascertain the properties of 
the newly discovered particle and includes results produced by the author for the Hadron Collider Physics (HCP)
symposium in November 2012.  

In addition to the work contained in this thesis, the author contributed towards early studies
in electroweak physics at CMS. The studies undertaken involved the development of a 
robust signal extraction technique used to measure the production cross-section of $W$ bosons,
via their decay to electrons, in proton-proton collisions at 7 TeV. 
The technique utilised control samples in data to subtract backgrounds from QCD, exploiting the kinematic
signature of the decay $W\rightarrow e\nu$. 
Re-establishing well measured Standard Model processes, such as $pp\rightarrow W\rightarrow e\nu$,
was one of the first major goals of CMS, ensuring a high level of understanding of the detector components
and their calibration. The analysis was performed on the first 36$fb^{-1}$ of 
data collected at CMS during 2010 and contributed towards the publication containing the $W$ cross-section
measurement from that dataset~\citep{EWK-11-001,AN-11-009}.  


